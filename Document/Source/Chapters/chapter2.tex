\chapter{Setting up and testing of a rendering library}

In this chapter the basis for the project is evaluated. In foresight at this 

not clear, whether the AR-Library will build on top of this library


\section{What WebGL is}
put to introduction

\section{Library for WebGL}
three.js vs babylon vs other vs one

\section{Setup web development environment}




(choose library in foresight to cardboard)
This chapter is mainly to understand the basics and see, what’s working
A look at some tutorials how to create an webGl engine shows, that it gets quite fast very complex.
So it is more to understand the concept and to see, whether the basics work



 

\section{Demo an evaluation}

\subsection*{WebGL Library}

At this state of the development the choose of a library is questionable in respect to use it as a base. WebGL is quite deep and complex concerning the distance to the user. What more, the library on top might not rely on the same library or just in a hidden way.
Undoubtedly it serves well 


\subsection*{used web services}

The tested 

They provide both a GitHub synchronization, which gives the possibility to backup code in a comfortable way.